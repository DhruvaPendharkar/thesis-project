\documentclass[]{article}
\usepackage{url}
\usepackage{hyperref}

%opening
\title{s(ASP) 1.0.5\\README}
\author{Kyle Marple (\url{kmarple1@hotmail.com)}}

\begin{document}

\maketitle


\tableofcontents


\section{Description}

The s(ASP) system is an implementation of the stable model semantics which does
not require any form of grounding. Unlike similar systems, it can work with
programs which have no finite grounding, as well as accept lists and terms as
predicate arguments.


\section{License}

The s(ASP) system is distributed under the 3-clause BSD license (a.k.a. BSD-3,
Modified BSD License or New BSD License). See the file COPYING for the full text
of the license.


\section{Package Contents}

This README should be part of a distribution containing the following files and
directories:

\begin{itemize}
	\item src/ -- Source code directory
	\item test/ -- Test directory containing several sample programs
	\item CHANGES -- Changelog
	\item COPYING -- 3-Clause BSD License
	\item INSTALL -- Dummy file, just directs the reader here
	\item Makefile -- Makefile for systems which support the 'make' command
	\item README -- This file
	\item README.pdf -- A PDF version of this file
	\item README.tex -- A latex version of this file
	\item winmake.bat -- A simple compilation script for Windows
\end{itemize}

\section{Installation}

Building s(ASP) requires SWI Prolog to be installed on the local machine. The
system has been tested using SWI Prolog version 7.x.x and may be incompatible
with older versions. Use of the 64-bit version of SWI Prolog is highly
recommended. The 32-bit version restricts memory allocation in such a way that
some s(ASP) programs may fail to run due to insufficient stack space. Finally,
the compilation methods below assume that it may be invoked using the `swipl'
command.

Prior to compiling s(ASP), you may wish to change the default settings for
execution mode, number of answer sets and stack size. This can be done by
editing the file `src/config.pl'. The various options are explained in the
comments of that file.

There are three methods for compiling s(ASP). On systems which support the
`make' command, the included Makefile can be used. On Windows systems without
access to the `make' command, the included `winmake.bat' may be used. Finally,
the following command can be executed from a command prompt in the `src/'
directory:
\begin{verbatim}
    swipl -L0 -G0 -T0 --goal=main --stand_alone=true -o sasp -c main.pl
\end{verbatim}
In all three cases, compilation will produce the executable `sasp',
which can then be moved as necessary.


\section{Usage}

The s(ASP) system accepts modified normal logic programs (see Input Program
Format below) and executes them under the stable model semantics. Two modes of
execution are supported: automatic and interactive.

Automatic mode will execute programs using either a hard-coded query or an empty
query if no hard-coded query is available. The number of stable models computed
will be determined by either the default value in `src/config.pl' or a value
given as a command line argument.

Interactive mode works similarly to traditional Prolog systems: the user is
provided with a prompt to enter queries. Answers can then be accepted by
pressing `.' or rejected by pressing `;'. Pressing `h' at the prompt will
explain both options.

The general format for invoking s(ASP) is:
\begin{verbatim}
    sasp [options] <inputfile(s)>
\end{verbatim}
The available options are:

\begin{tabular}{|c|l|}
	\hline 
	-h, -?, --help & Print this help message and terminate. \\ 
	\hline 
	-i, --interactive  & Run in user / interactive mode. \\ 
	\hline 
	-a, --auto & Run in automatic mode (no user interaction). \\ 
	\hline 
	-sN & Compute N answer sets, where N $\ge$ 0. 0 for all.\\
	& Ignored unless running in automatic mode. \\ 
	\hline 
	-v, --verbose & Enable verbose progress messages. \\ 
	\hline 
	-vv, --veryverbose & Enable very verbose progress messages. \\ 
	\hline 
	-j & Print proof tree for each solution. \\ 
	\hline 
	-n & Hide goals added to solution by global consistency\\
	& checks. \\ 
	\hline 
	-la & Print a separate list of succeeding abducibles with each\\
	& CHS. List will only be printed if at least one abducible\\
	& has succeeded. \\ 
	\hline 
\end{tabular}

\noindent For example,
\begin{verbatim}
    sasp -i test1.lp
\end{verbatim}
will execute the program `test1.lp' in interactive mode.


\section{Input Program Format}

The format for s(ASP) programs is that of normal logic programs (NLPs) with a
number of optional modifications. These modifications are hard-coded queries,
the ability to suppress specific predicates from output, a number of special
directive statements, negatively constrained variables and support for
arithmetic operations.


\subsection{Hard-Coded Queries}

Queries may be hard-coded into a program using one of the following two formats:
\begin{verbatim}
    ?- goal_1, ..., goal_n.
    #compute N { goal_1, ..., goal_n }.
\end{verbatim}
Both formats will execute the query `goal\_1, ..., goal\_n' when the program is
executed in automatic mode. In the second format, N is an integer specifying
the number of solutions to compute. A value of 0 indicates that all solutions
should be found. Queries using the first format will use the default solution
count specified in config.pl. For example, consider the following two queries:
\begin{verbatim}
    ?- member(X, [1, 2]).
    #compute 3 { member(X, [1,2]) }.
\end{verbatim}
The first will compute the default number of solutions, whatever that may be.
The second will attempt to compute 3 solutions. As there are only two (assuming
a standard definition of member/2), it will find the first two and then fail.

Several things are worth noting: First, the `-s' command line will override any
hard-coded solution count. Next, hard-coded queries are ignored when running in
interactive mode. Finally, if multiple queries are specified in a single
program, all but the last will be silently discarded.


\subsection{Hiding Predicates in Output}

As the models returned by s(ASP) can be quite large, it is often desirable to
omit unwanted predicates from the output. This can be done by appending an
underscore (`\_') character to the beginning of the predicate name. Such
predicates will be executed normally, but neither they nor their negations will
be included in any solutions. For example, consider the following program:
\begin{verbatim}
    p.
    q :- p.
    ?- q.
\end{verbatim}
When executed, the solution { p, q } will be printed. However, for the program
\begin{verbatim}
    p.
    _q :- p.
    ?- _q.
\end{verbatim}
only { p } will be printed.


\subsection{Directive Statements}

In addition to the \#compute statement discussed above, two other special
statements are supported by s(ASP): \#include and \#abducible.

The \#include statement is used to include additional source files, allowing a
program to be broken up across multiple files. Two formats are allowed:
\begin{verbatim}
    #include 'somefile.lp'.
\end{verbatim}
and
\begin{verbatim}
    #include('somefile.lp').
\end{verbatim}
If the file is in another directory, the relative or absolute path must also
be specified:
\begin{verbatim}
    #include 'somedir/somefile.lp'.
    #include './somedir/somefile.lp'.
    #include '../somedir/somefile.lp'.
    #include 'C:\somedirsomefile.lp'. % Windows-specific
\end{verbatim}
The \#abducible statement is included to simplify the use of s(ASP) for
abduction. Abducibles may be specified as follows:
    \#abducible some\_goal(X, Y).
This simply means that the specified goal may or may not be true. If encountered
during execution, it may be ``abduced'' even if the program contains no rules for
the goal.


\subsection{Negatively Constrained Variables}

In most logic programming systems, variables may be either bound or unbound. In
s(ASP), \textit{variables may be either bound or negatively constrained}. A negatively
constrained variable is associated with a \textbf{prohibited value set}, a list of ground
values which the variable cannot be bound to. Unbound variables are treated as a
special case of negatively constrained variables in which the prohibited value
set is empty.

Values are added to the prohibited value set via \textbf{disunification}. Consider the
following examples
\begin{verbatim}
    X \= 5, X \= a.
\end{verbatim}
Under most logic programming semantics, both of the above statements would fail,
since an unbound variable can be bound to any value. However, in s(ASP), both
statements will succeed, adding `5' and `a' to X's prohibited value list. Any
later attempt to bind X to one of these values will fail.

Note that, as explained in Section 7, two negatively constrained variables 
cannot be constrained against each other. Any attempt to do so will trigger an
error.


\subsection{Loop Variables}

Loop variables arise from a special case known as an \textbf{even loop}, in which a
recursive call is encountered with an even, non-zero number of negations between
the recursive call and its ancestor. An even loop indicates that every call in
the loop may be either true or false, unless some element of the cycle succeeds
or is forced to fail via other rules in the program.

If an even loop succeeds with the same negatively constrained variable in both 
the recursive call and its ancestor, this variable is known as a \textbf{loop variable}. 
In such cases, if the program has at least one stable model, it will have an 
infinite number of partial stable models. This is because predicates containing 
the loop variable may be true or false for every value in the loop variable's 
domain.

The s(ASP) system compactly represents these cases by prefixing loop variables
with a `?' in s(ASP)'s output. For example, consider the following program with
the query `?- p(X).': 
\begin{verbatim}
    p(X) :- not q(X).
    q(X) :- not p(X).
\end{verbatim}
An infinite number of partial stable models exist for this program, such as:
\begin{verbatim}
    { p(1), q(2), not p(2), not q(1) }
    { p(1), q(a), q(b), not p(a), not p(b), not q(1) }
\end{verbatim}
and so on. However, s(ASP) will produce only the following result:
\begin{verbatim}
    { p(?X), not q(?X) }
\end{verbatim}
This concisely represents that, for each value of ?X, p/1 and q/1 may be either
true or false, so long as they are not both true or both false for the same
value.


\subsection{Arithmetic and Comparisons}

The s(ASP) system provides a number of built-in, inline operators, described
below.

The two most important operators are unification (`=') and disunification
(`\textbackslash ='). For ground values, both operators function identically to their Prolog
equivalents. For one ground value and one negatively constrained variable,
unification will function as in Prolog, while disunification will add the ground
value to the variable's prohibited value list. For two negatively constrained
variables, unification will unify the variables and merge their prohibited value
lists, while disunification will produce an error (see Section 7).

The remaining arithmetic and comparison operators are detailed in table \ref{tab:operators}.

\begin{table}[htb]
	\begin{tabular}{|c|c|}
		\hline Operator & Usage \\ 
		\hline is & Arithmetic assignment; identical to Prolog operator \\ 
		\hline =:= & Equality (numbers) \\
		\hline =\textbackslash = & Inequality (numbers) \\
		\hline $<$ & Less than (numbers) \\
		\hline $=<$ & Less than or equal to (numbers) \\
		\hline $>$ & Greater than (numbers) \\
		\hline $>=$ & Greater than or equal to (numbers) \\
		\hline @$<$ & Less than (terms) \\
		\hline @$=<$ & Less than or equal to (terms) \\
		\hline @$>$ & Greater than (terms) \\
		\hline @$>=$ & Greater than or equal to (terms) \\
		\hline + & Addition \\
		\hline - & Subtraction \\
		\hline * & Multiplication \\
		\hline / & Division \\
		\hline rem & Remainder (integers only) \\
		\hline mod & Modulus (integers only) \\
		\hline $<<$ & Bitwise shift left (integers only): X is 8 $<<$ 1 yields X = 16. \\
		\hline $>>$ & Bitwise shift right (integers only): X is 8 $>>$ 1 yields X = 4. \\
		\hline $**$ & Exponent: X is 8**2 yields X = 64. \\
		\hline $\wedge$ & Exponent (identical to **) \\
		\hline 
	\end{tabular}
	\caption{s(ASP) Arithmetic Operators}
	\label{tab:operators}
\end{table}


\section{Restrictions on Input Programs}

In addition to the accepted program format, s(ASP) places three limitations on
what constitutes a `legal' program. First, as in Prolog, arithmetic statements
must be ground at the time they are evaluated. For example, the statement
\begin{verbatim}
    X is Y + 2.
\end{verbatim}
will succeed if Y is bound to a number, but will throw an error otherwise.
Negatively constrained variables can never be constrained against each other.
For example, given the query
\begin{verbatim}
    ?- X \= 2, Y \= 3, X \= Y.
\end{verbatim}
the final goal will throw an error. However, the query
\begin{verbatim}
    ?- X \= 2, Y is 3, X \= Y.
\end{verbatim}
will succeed because Y is no longer negatively constrained when it is disunified
with X. Finally, programs cannot allow success through left recursion. This
final restriction is a limitation of the current implementation, and we hope to
remove it in a future release.


\section{Applications Using s(ASP)}

Below are links to projects and papers which use s(ASP). If you would like your
project to be listed in future releases, please contact us using the email
address at the beginning of this file.

\begin{itemize}
\item s(ASP) Hackathon (\url{https://hackai16.devpost.com/submissions}) UT
Dallas's AI Society held a 24-hour s(ASP) hackathon on November 5-6, 2016. 18
projects were submitted by teams involving $\sim$60 UTD students.
\item  Zhuo Chen, Kyle Marple, Elmer Salazar, Gopal Gupta, Lakshman Tamil: A 
Physician Advisory System for Chronic Heart Failure Management Based on 
Knowledge Patterns. TPLP 16(5-6): 604-618 (2016) 
(\url{https://arxiv.org/abs/1610.08115})
\item Degree Audit Program (\url{https://gitlab.com/saikiran1096/gradaudit/}) 
A program to determine if a student has enough credits to graduate and display 
courses needed fulfill unmet requirements.
\end{itemize}

\end{document}
